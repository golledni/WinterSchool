%%%%%%%%%%%%%%%%%%%%%%%% ADD CONTENT %%%%%%%%%%%%%%%%%%%%%%

\section{Analysis and plotting of spatial data}
\begin{frame}
   \vspace{1cm}
    \Huge{
   FROM FILE  .....   \\
   \vspace{0.5cm}
   \hspace{5cm}  TO FIGURE ...  \\}
   \includegraphics[scale=0.5]{images/penguin-graph-cover.png}
\end{frame}

%*****************************************************************
% SEC0 | Outline
%*****************************************************************

\begin{frame}{\insertsectionnumber{ |} Outline}
   \begin{itemize}
       \item Downloaded a file, so what?
            \vspace{0.3cm}
       \item Play around with CDO
            \vspace{0.3cm}
       \item Load into Python and visualise
            \vspace{0.3cm}
       \item Make a publication figure
   \end{itemize}
\end{frame}

%*****************************************************************
% SEC1 | Downloaded a file, so what?
%*****************************************************************
\begin{frame}{\insertsectionnumber{ |} Downloaded a file, so what?}
\end{frame}

\begin{frame}{\insertsectionnumber{ |} Downloaded a file, so what? - Using \textbf{ncview}}
   \begin{itemize}
       \item What are the dimensions? what is the domain?
            \vspace{0.3cm}
       \item What are the variables?
            \vspace{0.3cm}
       \item What are the units? and range?
            \vspace{0.3cm}
       \item Is there masked data? what is the value used? 
            \vspace{0.3cm}
       \item How to create a quick timeseries?
   \end{itemize}
\end{frame}


\begin{frame}{\insertsectionnumber{ |} Downloaded a file, so what? - Using \textbf{ncview}}
    \includegraphics[scale=0.35]{images/wrf_RIS.png}\\
        \centering{WRF output, 2m temperature}
\end{frame}


% \begin{frame}{\insertsectionnumber{ |} Downloaded a file, so what? - Using \textbf{ncview}}
%     Using \textbf{ncview} to check the file:
%     \begin{columns}
%         \column[c]{6.5cm}
%             \centering\includegraphics[width=6cm]{images/Theta.png} \\
%                 \centering{ECCO2, potential temperature}
%         \column[c]{6.5cm}
%     \end{columns}
% \end{frame}


\begin{frame}{\insertsectionnumber{ |} Downloaded a file, so what? - Using \textbf{ncview}}
    \begin{columns}
        \column[c]{6.5cm}
            \centering\includegraphics[width=6cm]{images/Theta.png} \\
                \centering{ECCO2, potential temperature}
        \column[c]{6.5cm}
            \centering\includegraphics[width=6cm]{images/MIT_eccoV5.png} \\
                \centering{ECCO v5, potential temperature}
    \end{columns}
\end{frame}


\begin{frame}{\insertsectionnumber{ |} Downloaded a file, so what? - Using \textbf{ncview}}
    \flushleft\includegraphics[scale=0.33]{images/tripolar.png}\hspace*{1cm}\\
        \centering{IPSL-CM6A-LR, sea ice area fraction}
\end{frame}


\begin{frame}{\insertsectionnumber{ |} Downloaded a file, so what? - Using \textbf{panoply}}
    \centering\includegraphics[width=10cm]{images/Panoply1.png} \\
\end{frame}


\begin{frame}{\insertsectionnumber{ |} Downloaded a file, so what? - Using \textbf{panoply}}
    \centering\includegraphics[width=10cm]{images/Panoply2.png} \\
\end{frame}


\begin{frame}{\insertsectionnumber{ |} Downloaded a file, so what? - Using \textbf{panoply}}
    \centering\includegraphics[width=8.5cm]{images/Panoply3.png} \\
\end{frame}


\begin{frame}{\insertsectionnumber{ |} Downloaded a file, so what? - Using \textbf{panoply}}
    \centering\includegraphics[width=8.5cm]{images/Panoply4.png} \\
\end{frame}


\begin{frame}{\insertsectionnumber{ |} Downloaded a file, so what? - Using \textbf{panoply}}
    \centering\includegraphics[width=10cm]{images/Panoply5.png} \\
\end{frame}


\begin{frame}{\insertsectionnumber{ |} Downloaded a file, so what? - Using \textbf{panoply}}
    \centering\includegraphics[width=10cm]{images/Panoply6.png} \\
\end{frame}


\begin{frame}{\insertsectionnumber{ |} Downloaded a file, so what? - Using \textbf{panoply}}
    \centering\includegraphics[width=10cm]{images/Panoply7.png} \\
\end{frame}


\begin{frame}{\insertsectionnumber{ |} Downloaded a file, so what? - Using \textbf{panoply}}
    \centering\includegraphics[width=10cm]{images/Panoply8.png} \\
\end{frame}


\begin{frame}{\insertsectionnumber{ |} Downloaded a file, so what? - \textbf{coordinates}}
    Different coordinate systems:
        \vspace{0.3cm}
    \begin{itemize}
        \item lon-lat (easy to figure out where we are)
            \vspace{0.3cm}
        \item x-y (equidistant grid in km, better at the poles)
            \vspace{0.3cm}
        \item irregular grid (removes the pole problem)
    \end{itemize}
\end{frame}


\begin{frame}{\insertsectionnumber{ |} Downloaded a file, so what? - \textbf{coordinates}}
    \includegraphics[scale=0.25]{images/winter_school_domain_1.png}
\end{frame}


\begin{frame}{\insertsectionnumber{ |} Downloaded a file, so what? - \textbf{coordinates}}
    \includegraphics[scale=0.25]{images/winter_school_domain_2.png}
\end{frame}


\begin{frame}{\insertsectionnumber{ |} Downloaded a file, so what? - \textbf{coordinates}}
    \includegraphics[scale=0.25]{images/winter_school_domain_3.png}
\end{frame}

\begin{frame}{\insertsectionnumber{ |} Downloaded a file, so what? - \textbf{coordinates}}
    \includegraphics[scale=0.25]{images/winter_school_domain_3_2.png}
\end{frame}

\begin{frame}{\insertsectionnumber{ |} Downloaded a file, so what? - get information on the data}
    \begin{itemize}
        \item The naming \textbf{conventions} are useful: what do the two files contain?\\
            \vspace{0.5cm}
        \item Let's verify using \textbf{ncview} or \textbf{cdo}: \\
            \begin{itemize}
                \item \textit{ncdump filename} 
                \item \textit{ncdump -h filename} 
                \item \textit{cdo sinfo filename} 
            \end{itemize}
        \vspace{2cm}
    \begin{beamerboxesrounded}[lower=gray,shadow=true]{Pop quiz: open the theta....2013.nc file and:
        \begin{itemize}
            \item what is the variable(s)? short and long name?
            \item what is the spatial resolution of the data? and the dimension of the file?
            \item what is the temporal coverage of the data?
         \item is there a mask value anywhere? 
         \item how many dimensions does this variable have?
        \end{itemize}
        }
    \end{beamerboxesrounded}
    \end{itemize}
\end{frame}
 
 
%*****************************************************************
% SEC2 | Play around with CDO
%*****************************************************************
\begin{frame}{\insertsectionnumber{ |} Play around with \textbf{cdo}}
\end{frame} 


\begin{frame}{\insertsectionnumber{ |} Play around with \textbf{cdo}}
    Manipulating big files can be slow or make your program crash!\\
        \vspace{0.3cm}
    Pre-processing can save our life: let's use \href{http://www.idris.fr/media/ada/cdo.pdf}{\beamerbutton{\Huge{CDO}}}\\
        \vspace{0.5cm}
    \textbf{cdo} can be used to :
        \vspace{0.3cm}
    \begin{itemize}
        \item select
            \vspace{0.1cm}
            \begin{itemize}
                \item spatially
                    \vspace{0.1cm}
                \item temporally
                    \vspace{0.1cm}
                \item vertically
            \end{itemize}
        \item mask
        \item remap
        \item do statistics and calculations
        \item and so much more...
    \end{itemize}
\end{frame}     
        
\begin{frame}{\insertsectionnumber{ |} Play around with \textbf{cdo} - spatial extent}        
    To reduce the \textbf{spatial} extent:
    \vspace{0.3cm}
    \begin{itemize}
        \item \textit{cdo sellonlatbox,lon0,lon1,lat0,lat1 infile outfile }
           \vspace{0.3cm}
        \item \textit{cdo selindexbox,index0,index1,index0,index1 infile outfile }
    \end{itemize}
        \vspace{0.5cm}
    example: to select New Zealand (to be done on both theta and so files)\\
        \vspace{0.3cm}
    \textcolor{black}{cdo sellonlatbox,150,180,-30,-50 thetao\_Omon\_CESM2\_historical\_r1i1p1f1\_gr\_201301-201412.nc NZ\_theta.nc }\\
        \vspace{0.3cm}
\end{frame}
  
  
\begin{frame}{\insertsectionnumber{ |} Play around with \textbf{cdo} - temporal extent}
    To reduce the \textbf{temporal} extent:
        \vspace{0.3cm}
    \begin{itemize} 
        \item \textit{cdo selmon,1 infile outfile }
            \vspace{0.3cm}
        \item \textit{cdo select,timestep=1  infile outfile }
            \vspace{0.3cm}
         \item \textit{cdo splityear  infile prefix- }
    \end{itemize}
\end{frame}


\begin{frame}{\insertsectionnumber{ |} Play around with \textbf{cdo} - vertical (level) extent}
    To reduce the number of \textbf{levels}:\\
        \vspace{0.3cm}
    \begin{itemize}
        \item \textit{cdo sellevel,1 infile outfile }\\
            \vspace{0.3cm}
        \item \textit{cdo select,levrange=lev1,lev2,name=varname}\\
        \vspace{0.5cm}
    \end{itemize}
    example: let's select the second and third levels \\
        \vspace{0.3cm}
    \textcolor{black}{cdo select,level=10,20,name=thetao NZ\_theta.nc NZ\_theta\_levels.nc}\\
  \vspace{0.3cm}
        and let's do the same for salinity.
\end{frame}


\begin{frame}{\insertsectionnumber{ |} Play around with \textbf{cdo} - stats, remap and mask}
    But also to do stats, remap and mask data...\\
        \vspace{0.3cm}
    \begin{itemize}
        \item do some \textbf{stats}: 
            \vspace{0.2cm}
            \begin{itemize}
                \item \textit{cdo monmean ...}
                    \vspace{0.2cm}
                \item \textit{cdo ymonmean ...}
                    \vspace{0.2cm}
                \item \textit{cdo add(c) ...}, \textit{cdo sub(c) ...}
                    \vspace{0.2cm}            
                \item \textit{cdo monstd ...}
                    \vspace{0.2cm}
            \end{itemize}
        \item\textit{cdo remapbil ...} will use bilinear interpolation to \textbf{remap} one grid ont another (! create a grid description file using \textit{cdo griddes} beforehand)\\
            \vspace{0.5cm}
        \item\textit{cdo ifthenelse ...} will use a condition to determine if the \textbf{mask} is used (! create a mask file beforehand) \\
            \vspace{0.3cm}
            example: Let's mask out the areas with salinity > 34.5 for the temperature file:\\
                \vspace{0.3cm}
            \textcolor{black}{cdo mulc,0 NZ\_theta\_levels.nc zeroes.nc \\
            cdo ifthenelse -gec,34.5 NZ\_tso\_levels.nc zeroes.nc NZ\_theta\_levels.nc  test.nc}\\
                \vspace{0.5cm}
    \end{itemize}
\end{frame}


\begin{frame}{\insertsectionnumber{ |} Play around with \textbf{cdo} - pop quiz}
    \begin{beamerboxesrounded}[lower=gray,shadow=true]{Pop quiz: Use \textbf{cdo} commands to compare 
        \begin{itemize}
            \item two cross sections
                \vspace{0.15cm}
            \item of mean June
                \vspace{0.15cm}
            \item sea surface temperature
                \vspace{0.15cm}
            \item over the dateline
            \vspace{0.3cm}
            \item using the two thetao files (201301-201412 and 185001-185112)
                \vspace{0.15cm}
            \item using the red-blue colormap
                \vspace{0.15cm}
            \item with the surface at the top 
                \vspace{0.15cm}
            \item and a colorbar symmetrical around 0
        \end{itemize}
            \vspace{1.5cm}
        Hint: subtract one from the other}
    \end{beamerboxesrounded}
\end{frame}


%*****************************************************************
% SEC3 | Load into Python and visualise
%*****************************************************************
\begin{frame}{\insertsectionnumber{ |} Load into Python and visualise}
\end{frame} 


\begin{frame}{\insertsectionnumber{ |} Load into Python and visualise - essentials}
    Now, time to create a script to make reproducible plots:
        \vspace{0.3cm}
    \begin{itemize}
        \item let's create a python script (\textit{decide\_on\_a\_relevant\_and\_clear\_name\textbf{.py}})
             \vspace{0.3cm}
         \item one figure per file is a useful rule
             \vspace{0.3cm}
        \item add comments to explain what we do
             \vspace{0.3cm}
        \item prepare our data before loading it
             \vspace{0.3cm}
        \item to run a python script : \textit{python scriptname.py} in the terminal
            \vspace{0.3cm}
    \end{itemize}
\end{frame}
 
 
\begin{frame}{\insertsectionnumber{ |} Load into Python and visualise - packages}
    Let's import the relevant python packages:\\
        \vspace{0.3cm}
    \hbox{\hspace{-0.8cm}\includegraphics[scale=0.35]{images/Script1_step1.png}}
\end{frame}
 
 
\begin{frame}{\insertsectionnumber{ |} Load into Python and visualise - read a netCDF} 
    Let's import the netcdf dataset and get a sense of the dimensions:\\
        \vspace{0.5cm}
    \includegraphics[scale=0.35]{images/Script1_step2.png}
        \vspace{2cm}
    (24, 33, 180, 360)
    \begin{beamerboxesrounded}[lower=gray,shadow=true]{Pop quiz:\\
       To which dimensions do these numbers correspond?}
    \end{beamerboxesrounded}
\end{frame}
 
 
\begin{frame}{\insertsectionnumber{ |} Load into Python and visualise - read a netCDF}
    Let's get the dimensions we want and plot the data:\\
        \vspace{0.5cm}
    we want to look at the surface, first time step and whole spatial extent of the file
        \vspace{0.5cm}
    \includegraphics[scale=0.35]{images/Script1_step3.png}
\end{frame}
 
 
\begin{frame}{\insertsectionnumber{ |} Load into Python and visualise - simple plot} 
    \includegraphics[scale=0.45]{images/script1_fig1.png}
\end{frame}
 
  
\begin{frame}{\insertsectionnumber{ |} Load into Python and visualise - simple plot} 
    Let's add the latitude and longitude coordinates:\\
        \vspace{0.5cm}
    \includegraphics[scale=0.35]{images/Script1_step4.png}
\end{frame}
  
  
\begin{frame}{\insertsectionnumber{ |} Load into Python and visualise - simple plot} 
    \includegraphics[scale=0.45]{images/Script1_fig2.png}
\end{frame}
 
 
\begin{frame}{\insertsectionnumber{ |} Load into Python and visualise - colormap} 
    \includegraphics[scale=0.35]{images/Nature.png}\\
        \vspace{0.5cm}
     Let's add a *fancy* colormap:\\
    \includegraphics[scale=0.35]{images/Script1_step5.png}\\
        \vspace{0.3cm}
    (!! we need to add the batlow.py to the working directory)\\
\end{frame}
 
 
\begin{frame}{\insertsectionnumber{ |} Load into Python and visualise - colormap} 
    \vspace{0.5cm}
    \includegraphics[scale=0.45]{images/Script1_fig3.png}
\end{frame}
 
 
\begin{frame}{\insertsectionnumber{ |} Load into Python and visualise - colormap} 
    \vspace{0.5cm}
    \includegraphics[scale=0.20]{images/Colormap_1.png}
\end{frame}
 
 
\begin{frame}{\insertsectionnumber{ |} Load into Python and visualise - colormap}
    \vspace{0.5cm}
    \includegraphics[scale=0.20]{images/Colormap_2.png}
\end{frame}
  
\begin{frame}{\textbf{3 |} Load into Python and visualise - colormap} 
    \vspace{0.5cm}
    \includegraphics[scale=0.20]{images/Colormap_3.png}
\end{frame}
  
  
 
\begin{frame}{\insertsectionnumber{ |} Load into Python and visualise - labels and colorbar} 
    Let's add labels and a colorbar limited at 20$^{\circ}$C, but pay attention to the variable names and units!
        \vspace{0.5cm}
    \includegraphics[scale=0.35]{images/Script1_step6.png}
\end{frame}
 
 
\begin{frame}{\insertsectionnumber{ |} Load into Python and visualise - labels and colorbar} 
        \vspace{0.5cm}
    \includegraphics[scale=0.45]{images/Script1_fig4.png}
\end{frame}
 
 
\begin{frame}{\insertsectionnumber{ |} Load into Python and visualise - masking} 
    And, if we wanted to mask out some data, we can use \textbf{cdo} as done previously\\
    But it is also possible in python:\\
        \vspace{0.2cm}
    To mask out a certain value:\\
    \includegraphics[scale=0.35]{images/Script1_step7.png}\\
        \vspace{0.5cm}
    and to mask an area based on lon/lat:
        \vspace{0.2cm}
    lon and lat are 1D variables, we need to make them 2D first:
    \includegraphics[scale=0.35]{images/Script1_step8.png}  
        \vspace{0.3cm}
\end{frame}
 
  
\begin{frame}{\insertsectionnumber{ |} Load into Python and visualise - masking} 
    \begin{columns}
        \column[c]{6.5cm}
        To mask out a certain value:
            \vspace{0.5cm}
        \centering\includegraphics[scale=0.25]{images/Script1_fig5.png}
            \vspace{0.5cm}
        \column[c]{6.5cm}
        and to mask an area based on lon/lat:
            \vspace{0.3cm}
        \includegraphics[scale=0.25]{images/Script1_fig6.png}
            \vspace{0.3cm}
    \end{columns}
\end{frame}

%###################################################################
 
\begin{frame}{\insertsectionnumber{ |} Load into Python and visualise - cross-sections}
    We swap dimensions and plot along two dimensions (lon/time, depth/time,...).\\
        \includegraphics[scale=0.35]{images/Script2_step1.png}
\end{frame}
          
          
\begin{frame}{\insertsectionnumber{ |} Load into Python and visualise - cross-sections}
    \includegraphics[scale=0.45]{images/Script1_fig7.png}
\end{frame}
 
  
\begin{frame}{\insertsectionnumber{ |} Load into Python and visualise - cross-sections}
    To reverse the plot:\\
    \includegraphics[scale=0.35]{images/Script2_step2.png}\\
    And add labels, colorbar,...
    \includegraphics[scale=0.35]{images/Script2_step3.png}
\end{frame}
 
 
\begin{frame}{\insertsectionnumber{ |} Load into Python and visualise - cross-sections} 
    \includegraphics[scale=0.45]{images/script1_fig8.png}
\end{frame}


\begin{frame}{\insertsectionnumber{ |} Load into Python and visualise - cross-sections}
    And finally add contour lines:\\
    \includegraphics[scale=0.35]{images/Script2_step4.png}\\
\end{frame}


\begin{frame}{\insertsectionnumber{ |} Load into Python and visualise - cross-sections}
    \includegraphics[scale=0.45]{images/script1_fig9.png}
\end{frame}

% ####################################################################

\begin{frame}{\insertsectionnumber{ |} Load into Python and visualise - time-depth diagram} 
    Let's plot one location over time, with depth: \\
    \hbox{\hspace{-0.5cm}\includegraphics[scale=0.35]{images/Script3_step1.png}}
\end{frame}


\begin{frame}{\insertsectionnumber{ |} Load into Python and visualise - time-depth diagram} 
    Let's plot one location over time, with depth: \\
    \hbox{\hspace{-0.5cm}\includegraphics[scale=0.35]{images/Script3_step2.png}}
\end{frame}


\begin{frame}{\insertsectionnumber{ |} Load into Python and visualise - time-depth diagram} 
    \includegraphics[scale=0.45]{images/script1_fig10.png}
\end{frame}


\begin{frame}{\insertsectionnumber{ |} Load into Python and visualise - pop quiz}
    \begin{beamerboxesrounded}[lower=gray,shadow=true]{Pop quiz:\\
        \begin{itemize}
            \item Try to find another location (different hemisphere or equator),\\
            or change the depth range on the y-axis.
            \item Do an horizontal section along the equator / 30$^{\circ}$S with depth values. \\
            Compare the extent of the mixed layers
        \end{itemize}}
    \end{beamerboxesrounded}
\end{frame} 

%*****************************************************************
% SEC4 |Make a publication figure
%*****************************************************************

\begin{frame}{\insertsectionnumber{ |} Make a publication figure}
    Now that we have a pretty figure, let's start being fancy.\\
        \vspace{0.3cm}
%    Mapping is the \textbf{second} step, not the first!\\
%        \vspace{0.5cm}
    We will use the \href{https://scitools.org.uk/cartopy/docs/latest/}{\beamerbutton{\Huge{cartopy}}} package:\\
    \includegraphics[scale=0.35]{images/Script4_step1.png}
\end{frame}


\begin{frame}{\insertsectionnumber{ |} Make a publication figure - cartopy}
    Let's plot SST at the surface, over the whole domain with the 'Plate Carree" projection
        \includegraphics[scale=0.35]{images/Script4_step2.png}
\end{frame}
  
  
\begin{frame}{\insertsectionnumber{ |} Make a publication figure - cartopy} 
    \includegraphics[scale=0.50]{images/script2_fig1.png}
\end{frame}


\begin{frame}{\insertsectionnumber{ |} Make a publication figure - cartopy} 
    Let's fill in the variable, but lat and lon have to be  2D (use meshgrid)
    \includegraphics[scale=0.35]{images/Script4_step3.png}
\end{frame}


\begin{frame}{\insertsectionnumber{ |} Make a publication figure - cartopy} 
    \includegraphics[scale=0.50]{images/script2_fig2.png}
\end{frame}
 
 
\begin{frame}{\insertsectionnumber{ |} Make a publication figure - cartopy} 
    Let's format the axes:
    \includegraphics[scale=0.35]{images/Script4_step4.png}
\end{frame}


\begin{frame}{\insertsectionnumber{ |} Make a publication figure - cartopy} 
    \includegraphics[scale=0.50]{images/script2_fig3.png}
\end{frame}


\begin{frame}{\insertsectionnumber{ |} Make a publication figure - cartopy}
    Imagine: \\
    \begin{itemize}
        \item we have a CTD station (at 60$^{\circ}$CW 40$^{\circ}$CS and 4500 m depth (30th level)) 
            \vspace{0.3cm}
        \item and let's add the colorbar
    \end{itemize}
    \includegraphics[scale=0.35]{images/Script4_step5.png}
\end{frame}


\begin{frame}{\insertsectionnumber{ |} Make a publication figure - cartopy}
    \includegraphics[scale=0.60]{images/script2_fig4.png}
\end{frame}

%######## xarray example

\begin{frame}{\insertsectionnumber{ |} Make a publication figure - \textbf{xarray}}
    \href{https://xarray.pydata.org/}{\beamerbutton{\Huge{xarray}}} is an useful package when dealing with netCDFs containing metadata: \\
        \vspace{0.5cm}
    \includegraphics[scale=0.35]{images/xarray_info.png}
\end{frame}


\begin{frame}{\insertsectionnumber{ |} Make a publication figure - \textbf{xarray}}
    \includegraphics[scale=0.45]{images/xarray_1.png}
\end{frame}


\begin{frame}{\insertsectionnumber{ |} Make a publication figure - \textbf{xarray}}
    \includegraphics[scale=0.35]{images/print_xarray.png}
\end{frame}


\begin{frame}{\insertsectionnumber{ |} Make a publication figure - \textbf{xarray}}
    \includegraphics[scale=0.35]{images/xarray_2.png}
\end{frame}

\begin{frame}{\insertsectionnumber{ |} Make a publication figure - \textbf{xarray}}
The title, colorbar,.. are automatically added:\\
    \includegraphics[scale=0.50]{images/figure_xarray.png}
\end{frame}

%###########
\begin{frame}{\insertsectionnumber{ |} Make a publication figure - difference between two datasets in \textbf{cdo}}
    What if we want to compare two datasets, which are not on the same grid? \\
        \vspace{0.5cm}
    Re-Mapping is the answer !\\
        \vspace{0.5cm}
    We can do it in \textbf{cdo}: \\
    \textit{cdo remapbil ...} will use bilinear interpolation to remap one grid ont another \\
    (! create a grid description file using \textit{cdo griddes} beforehand)\\
\end{frame}


\begin{frame}{\insertsectionnumber{ |} Make a publication figure - difference between two datasets in \textbf{cdo}}
    Get the grid description of the file you want to remap \textbf{onto}: (here, ERA5)\\
        \vspace{0.5cm}
    \textcolor{black}{cdo griddes SST\_ERA5\_201301.nc > grid\_era5.txt}\\
        \vspace{0.5cm}
    \includegraphics[scale=0.35]{images/Script5_step0.png}
\end{frame}
  
  
\begin{frame}{\insertsectionnumber{ |} Make a publication figure - difference between two datasets in \textbf{cdo}} 
    Then: 
        \begin{itemize}
            \item we remap using a bilinear interpolation: \\
                \textcolor{black}{cdo remapbil,grid\_era5.txt thetao\_Omon\_CESM2\_historical\_r1i1p1f1\_gr\_201301-201412.nc theta\_remapped.nc}
                    \vspace{0.3cm}
            \item  to subtract one from the other, the two files have to have the 'same variable' (name):\\
                \textcolor{black}{cdo chname,thetao,sst theta\_remapped.nc theta\_remapped\_sst.nc} \\
                    \vspace{0.3cm}
            \item and subtract the two files:\\
                \textcolor{black}{cdo sub SST\_ERA5\_201301.nc theta\_remapped\_sst.nc diff\_era5\_theta.nc} \\
                    \vspace{0.3cm} 
            \item let's check the nc file, and play with the colorbar, range,...
        \end{itemize}
\end{frame}

  
\begin{frame}{\insertsectionnumber{ |} Make a publication figure - remap with griddata} 
    Now, let's do the same in Python !\\
        \vspace{0.3cm} 
    We will use the \textbf{griddata package}\\
    \includegraphics[scale=0.35]{images/Script5_step1.png}
\end{frame}


\begin{frame}{\insertsectionnumber{ |} Make a publication figure - remap with griddata} 
    And it needs the data to be formatted in a specific way:
    \includegraphics[scale=0.35]{images/Script5_step2.png}
\end{frame}


\begin{frame}{\insertsectionnumber{ |} Make a publication figure - remap with griddata}
    And it needs the data to be formatted in a specific way:
    \includegraphics[scale=0.35]{images/Script5_step3.png}
\end{frame}


\begin{frame}{\insertsectionnumber{ |} Make a publication figure - make different subplots} 
    Let's make 3 subplots in the figure: 
    \begin{itemize}
        \item the ERA5 data
        \item the CMIP regridded data
        \item the difference between the two
    \end{itemize}
        \vspace{0.5cm}
    \includegraphics[scale=0.35]{images/Script5_step4.png}
\end{frame}


\begin{frame}{\insertsectionnumber{ |} Make a publication figure - make different subplots} 
    \includegraphics[scale=0.55]{images/script5_fig2.png}
        \vspace{0.3cm}
\end{frame}


\begin{frame}{\insertsectionnumber{ |} Make a publication figure - make different subplots} 
    Add the colorbars : \\
    \begin{itemize}
        \item batlow, but same limits as the second subplot
            \vspace{0.3cm}
        \item batlow, but same limits as the first subplot
            \vspace{0.3cm}
        \item red-blue (the vik colorbar)!
    \end{itemize}
    \vspace{0.3cm}
    Hint: we can print out the min and max values of our variables :
    \includegraphics[scale=0.35]{images/hint.png}\\
\end{frame}

\begin{frame}{\insertsectionnumber{ |} Make a publication figure - make different subplots} 
    \includegraphics[scale=0.35]{images/Script5_step5.png}
\end{frame}


\begin{frame}{\insertsectionnumber{ |} Make a publication figure - make different subplots} 
    And to center the vik colorbar around 0 : \\
        \vspace{0.3cm}
    \includegraphics[scale=0.35]{images/Script5_step6.png}
\end{frame}


\begin{frame}{\insertsectionnumber{ |} Make a publication figure - make different subplots} 
    And to center the vik colorbar around 0: \\
        \vspace{0.3cm}
    (hint: find out the min and max values) \\
        \vspace{0.3cm}
    \includegraphics[scale=0.35]{images/Script5_step7.png}
\end{frame}


\begin{frame}{\insertsectionnumber{ |} Make a publication figure - make different subplots} 
    \centering\includegraphics[scale=0.25]{images/script5_fig3.png}
    \begin{beamerboxesrounded}[lower=gray,shadow=true]{
        Pop quiz:\\
        Our figure will \textbf{not} look like this, find out why and fix it!}
    \end{beamerboxesrounded}
\end{frame}


\begin{frame}{\insertsectionnumber{ |} Make a publication figure - save the figure} 
    Finally, let's set the layout and save the figure: \\
        \vspace{0.3cm}
    \includegraphics[scale=0.35]{images/Script5_step8.png}
\end{frame}


\begin{frame}{\insertsectionnumber{ |} Make a publication figure - pop quiz} 
    \begin{beamerboxesrounded}[lower=gray,shadow=true]{
        Pop quiz: \\
            \vspace{0.3cm}
        Let's do the same as for the final \textit{cdo} command, but in a python script:\\
            \begin{itemize}
                \item two cross sections
                    \vspace{0.15cm}
                \item of mean June
                    \vspace{0.15cm}
                \item sea surface temperature
                    \vspace{0.15cm}
                \item over the dateline
                    \vspace{0.3cm}
                \item using the two thetao files (201301-201412 and 185001-185112)
                    \vspace{0.15cm}
                \item using the red-blue colormap
                    \vspace{0.15cm}
                \item with the surface at the top 
                    \vspace{0.15cm}
                \item and a colorbar symmetrical around 0
            \end{itemize}
                \vspace{1.5cm}}
    \end{beamerboxesrounded}
\end{frame}


\begin{frame}{\insertsectionnumber{ |} Make a publication figure - pop quiz} 
    \centering\includegraphics[scale=0.22]{images/script6_fig1.png}\\
\end{frame}


\begin{frame}{\insertsectionnumber{ |} Make a publication figure - pop quiz}
    \begin{columns}
        \column[c]{5.5cm}
            And as a final check,\\
                \vspace{0.3cm}
             load your cdo final result and \\
                \vspace{0.3cm}
             compare to the output of this script! 
        \column[c]{6.5cm}             
            \includegraphics[scale=0.22]{images/script6_fig2.png}\\ 
    \end{columns}
\end{frame}

%%%%%%%%%%%%%%%%%%%%%%%% END CONTENT %%%%%%%%%%%%%%%%%%%%%%