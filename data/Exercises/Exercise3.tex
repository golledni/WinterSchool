%%%%%%%%%%%%%%%%%%%%%%%%%%%%%%%%%%%%%%%%%%%%%%%%%%%%%%%%%%%%%%%
%
% Welcome to Overleaf --- just edit your LaTeX on the left,
% and we'll compile it for you on the right. If you open the
% 'Share' menu, you can invite other users to edit at the same
% time. See www.overleaf.com/learn for more info. Enjoy!
%
%%%%%%%%%%%%%%%%%%%%%%%%%%%%%%%%%%%%%%%%%%%%%%%%%%%%%%%%%%%%%%%
\documentclass{article}
\begin{document}

Shelf seas and marine biogeochemistry

Angela

Abstract
Shelf seas are regions of high biological activity, contributing 15-30\% of global oceanic primary production, with temperate shelf seas as an important global carbon sink. To understand how shelf seas will respond to environmental changes it is important to fully understand phytoplankton dynamics and inter-annual variability of phytoplankton production in these areas. Previous modelling works have shown that meteorology can affect phytoplankton seasonal dynamics but there is still debate in the literature about the direct mechanisms that affect long-term phytoplankton productivity. 


1 Introduction 

Shelf seas are ocean regions where water depth is less than a few hundred metres ($\sim 200$ m). They are separated from the deep ocean by a shelf break, where the seabed inclination generally increases rapidly from the top of the continental slope to the abyssal ocean. In these regions, the effects of friction and boundaries play a crucial role in determining ocean dynamics, experiencing a physical regime which is distinct from that of the abyssal ocean where depths are measured in kilometres. 

The rest of this paper is organized as follows. In section 2, we describe the methodology used. In section 3, we conclude.

2 Method

The process that allows the development of stratification can be defined by the potential energy anomaly (PEA, $\Phi$). This parameter is a quantitative measure of stratification and represents the work required per unit volume to completely mix the water column $[Jm^{-3}]$. \\

   $  Phi=frac{1}{h}int_{-h}^{0}({rho}-rho(z))gzdz $

where z is the water column depth (m), h is the mixed layer thickness (m), g=9.81 (m $\rm s^{-2}$) is the gravitational acceleration, $\bar{\rho}$ is the water column mean density determined from temperature and salinity profiles, $\rho(z)$ is the density profile determined from temperature and salinity 
profiles. 

3 Conclusions

There are different ways to study shelf seas including using research vessels and remote, autonomous vehicles. Water samples are needed to understand the ecology and biogeochemistry of the system. However, data from research vessels is limited as it is not synoptic, i.e. it is not sampled at different locations simultaneously and, because of this, remote sensing plays an important role in the study of shelf seas. With satellite data it is possible to obtain information about the distribution of net phytoplankton production in the ocean, but other phenomena such as the SCM are difficult to observe. Besides, suspended sediments and dissolved organic material can influence this data. To complement the available data from research vessels and remote sensing, ocean models are used to study and understand marine biogeochemistry. 


\end{document}